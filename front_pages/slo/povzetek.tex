%---------------------------------------------------------------
% SLO: slovenski povzetek
% ENG: slovenian abstract
%---------------------------------------------------------------
\selectlanguage{slovene} % Preklopi na slovenski jezik
\addcontentsline{toc}{chapter}{Povzetek}
\chapter*{Povzetek}

\noindent\textbf{Naslov:} \ttitle
\bigskip

Barvna fotografija je prišla v vsakdanjo uporabo šele v zadnjih 50 letih, zato imajo naši stari starši še vedno mnogo črno-belih fotografij, katere radi pogledajo in obujajo spomine. Bolj prijetno je gledati na fotografije, ki imajo prisotnih saj nekaj barv. Te obenem tudi bolj naravno prikažejo objekte in ljudi na sliki. To je razlog za razvoj različnih pristopov za barvanje črno-belih fotografij in videov. 

V tem delu predstavljamo nove avtomatske pristope za barvanje slik po delih, ki bazirajo na konvolucijskih nevronskih mrežah. Te pristopi nam omogočajo barvanje slik želene velikosti skoraj z enako natančnostjo kot tiste na katerih je bila mreža učena. V delu predstavljamo tudi primerjavo naših pristopov z pristopih iz sorodnih del. 

\subsection*{Ključne besede}
\textit{\tkeywords}
\clearemptydoublepage