%---------------------------------------------------------------
% SLO: slovenski povzetek
% ENG: slovenian abstract
%---------------------------------------------------------------
\selectlanguage{slovene} % Preklopi na slovenski jezik
\addcontentsline{toc}{chapter}{Povzetek}
\chapter*{Povzetek}

\noindent\textbf{Naslov:} \ttitle
\bigskip

Barvna fotografija je prišla v vsakdanjo uporabo šele v zadnjih 50 letih, zato imajo naši stari starši še vedno mnogo črno-belih fotografij, katere bi radi obarvali. V ta namen so bili razviti različni polavtomatski in avtomatski pristopi. 

V tem delu predstavljamo nekaj novih avtomatskih pristopov za barvanje črno-belih slik in videov, ki bazirajo na konvolucijskih nevronskih mrežah. Predstavljamo pristope po delih, ki nam omogočajo barvanje slik želene velikosti skoraj z enako natančnostjo kot tiste na katerih je bila mreža naučena. V delu so pristopi podrobneje analizirani in primerjani s pristopi iz sorodnih del. Preizkusili smo jih tudi na starih črno-belih slikah in videu. 

\subsection*{Ključne besede}
\textit{\tkeywords}
\clearemptydoublepage