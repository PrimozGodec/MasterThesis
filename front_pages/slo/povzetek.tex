%---------------------------------------------------------------
% SLO: slovenski povzetek
% ENG: slovenian abstract
%---------------------------------------------------------------
\selectlanguage{slovene} % Preklopi na slovenski jezik
\addcontentsline{toc}{chapter}{Povzetek}
\chapter*{Povzetek}

%\noindent\textbf{Naslov:} \ttitle
%\bigskip

Barvna fotografija je prišla v vsakdanjo uporabo šele v zadnjih 50 letih, zato so razni arhivi polni črno-belih fotografij, katere bi njihovi lastniki radi obarvali. V ta namen so bili razviti različni algoritmični pristopi.
V disertaciji predstavljamo nekaj novih avtomatskih pristopov za barvanje črno-belih slik in videov, ki so osnovani na strojnem učenju in konvolucijskih nevronskih mrežah. Pristope primerjamo s pristopi iz sorodnih del in jih preizkusimo na starih črno-belih slikah. 
Iz rezultatov je razvidno, da naši pristopi dosegajo kvaliteto barvanja pristopov iz sorodnih del. Naš nov pristop, ki obarva slike po delih, pa izboljša barvanje slik velikosti, ki so različne od tistih, na katerih je bila mreža naučena. Ta pristop je tudi naučen hitreje kot obstoječi pristopi, ki za barvanje uporabljajo celotne slike. 

\subsection*{Ključne besede}
\textit{\tkeywords}
\clearemptydoublepage